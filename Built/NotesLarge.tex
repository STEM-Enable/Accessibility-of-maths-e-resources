% Options for packages loaded elsewhere
\PassOptionsToPackage{unicode}{hyperref}
\PassOptionsToPackage{hyphens}{url}
%
\documentclass[
  17pt,
  english,
  a4paper]{extarticle}
\usepackage{lmodern}
\usepackage{amssymb,amsmath}
\usepackage{ifxetex,ifluatex}
\ifnum 0\ifxetex 1\fi\ifluatex 1\fi=0 % if pdftex
  \usepackage[T1]{fontenc}
  \usepackage[utf8]{inputenc}
  \usepackage{textcomp} % provide euro and other symbols
\else % if luatex or xetex
  \usepackage{unicode-math}
  \defaultfontfeatures{Scale=MatchLowercase}
  \defaultfontfeatures[\rmfamily]{Ligatures=TeX,Scale=1}
\fi
% Use upquote if available, for straight quotes in verbatim environments
\IfFileExists{upquote.sty}{\usepackage{upquote}}{}
\IfFileExists{microtype.sty}{% use microtype if available
  \usepackage[]{microtype}
  \UseMicrotypeSet[protrusion]{basicmath} % disable protrusion for tt fonts
}{}
\makeatletter
\@ifundefined{KOMAClassName}{% if non-KOMA class
  \IfFileExists{parskip.sty}{%
    \usepackage{parskip}
  }{% else
    \setlength{\parindent}{0pt}
    \setlength{\parskip}{6pt plus 2pt minus 1pt}}
}{% if KOMA class
  \KOMAoptions{parskip=half}}
\makeatother
\usepackage{xcolor}
\IfFileExists{xurl.sty}{\usepackage{xurl}}{} % add URL line breaks if available
\IfFileExists{bookmark.sty}{\usepackage{bookmark}}{\usepackage{hyperref}}
\hypersetup{
  pdftitle={Accessible maths e-resources},
  pdfauthor={Emma Cliffe, Skills Centre: MASH, University of Bath},
  pdflang={en},
  hidelinks,
  pdfcreator={LaTeX via pandoc}}
\urlstyle{same} % disable monospaced font for URLs
\usepackage[margin=2.5cm]{geometry}
\usepackage{longtable,booktabs}
% Correct order of tables after \paragraph or \subparagraph
\usepackage{etoolbox}
\makeatletter
\patchcmd\longtable{\par}{\if@noskipsec\mbox{}\fi\par}{}{}
\makeatother
% Allow footnotes in longtable head/foot
\IfFileExists{footnotehyper.sty}{\usepackage{footnotehyper}}{\usepackage{footnote}}
\makesavenoteenv{longtable}
\usepackage{graphicx,grffile}
\makeatletter
\def\maxwidth{\ifdim\Gin@nat@width>\linewidth\linewidth\else\Gin@nat@width\fi}
\def\maxheight{\ifdim\Gin@nat@height>\textheight\textheight\else\Gin@nat@height\fi}
\makeatother
% Scale images if necessary, so that they will not overflow the page
% margins by default, and it is still possible to overwrite the defaults
% using explicit options in \includegraphics[width, height, ...]{}
\setkeys{Gin}{width=\maxwidth,height=\maxheight,keepaspectratio}
% Set default figure placement to htbp
\makeatletter
\def\fps@figure{htbp}
\makeatother
\setlength{\emergencystretch}{3em} % prevent overfull lines
\providecommand{\tightlist}{%
  \setlength{\itemsep}{0pt}\setlength{\parskip}{0pt}}
\setcounter{secnumdepth}{5}
\ifxetex
  % Load polyglossia as late as possible: uses bidi with RTL langages (e.g. Hebrew, Arabic)
  \usepackage{polyglossia}
  \setmainlanguage[]{english}
\else
  \usepackage[shorthands=off,main=english]{babel}
\fi

\title{Accessible maths e-resources}
\author{Emma Cliffe, Skills Centre: MASH, University of Bath}
\date{TALMO, June 2020}

\usepackage{amsthm}
\theoremstyle{plain}
\newtheorem{theorem}{Theorem}[section]
\newtheorem{Thought}{Thought}[section]
\theoremstyle{plain}
\newtheorem{proposition}[theorem]{Proposition}
\newtheorem{Nugget}[Thought]{Nugget}
\theoremstyle{plain}
\newtheorem{lemma}{Lemma}[section]
\theoremstyle{plain}
\newtheorem{corollary}{Corollary}[section]
\theoremstyle{plain}
\newtheorem{conjecture}{Conjecture}[section]
\theoremstyle{definition}
\newtheorem{definition}{Definition}[section]
\theoremstyle{definition}
\newtheorem{example}{Example}[section]
\theoremstyle{definition}
\newtheorem{exercise}{Exercise}[section]
\theoremstyle{remark}
\newtheorem*{remark}{Remark}
\newtheorem*{solution}{Solution}
\let\BeginKnitrBlock\begin \let\EndKnitrBlock\end


%\usepackage[english,shorthands=off]{babel}
\usepackage{spverbatim}
\renewcommand{\familydefault}{phv}
\fontfamily{phv}\selectfont
\renewcommand{\em}{\bf}\renewcommand{\textit}{\textbf}\renewcommand{\emph}{\textbf}\renewcommand{\it}{\bf}\renewcommand{\itshape}{\bf}
\setlength{\parindent}{0.0pt}
\setlength{\parskip}{1.0\baselineskip}
\renewcommand{\baselinestretch}{1.5}\selectfont
\setlength{\mathsurround}{0.2em}
\setlength{\arraycolsep}{0.5cm}\renewcommand{\arraystretch}{1.5}
\addtolength{\jot}{\baselineskip}
\renewcommand{\;}{\,}
\sloppy
\allowdisplaybreaks
\newtheoremstyle{plain}{20pt}{3pt}{}{}{\bfseries}{.\newline\nobreak}{1.0em\nobreak}{}
\newtheoremstyle{definition}{20pt}{3pt}{}{}{\bfseries}{.\newline\nobreak}{1.0em\nobreak}{}
\newtheoremstyle{remark}{20pt}{3pt}{}{}{\bfseries}{.\newline\nobreak}{1.0em\nobreak}{}
\newenvironment{Proof}
  {\noindent{\bf Proof.}\hspace*{1em}}% Begin
  {\qed\par}% End
%% When redefining an environment it is vital that it has 
%% the same number of arguments as the original
\renewenvironment{proof}[1][\proofname]
  {\trivlist\item\relax\noindent{\bf {#1}.}\hspace*{1em}}% Begin
  {\qed\endtrivlist}% End

\begin{document}
\maketitle

{
\setcounter{tocdepth}{2}
\tableofcontents
}
\newpage
\pagenumbering{arabic}

\hypertarget{where-do-you-start}{%
\section*{Where do you start?}\label{where-do-you-start}}
\addcontentsline{toc}{section}{Where do you start?}

\begin{itemize}
\tightlist
\item
  Why is it important?
\item
  What do you need to do?
\item
  How do you get started?
\end{itemize}

\hypertarget{why-is-it-important}{%
\section{Why is it important?}\label{why-is-it-important}}

\begin{itemize}
\tightlist
\item
  New law
\item
  Processing of concepts - copy across
\item
  All students may need to work with text in new ways in the new world e.g.~Using their ebook reader to get away from their laptop
\end{itemize}

\hypertarget{what-do-you-need-to-do}{%
\section{What do you need to do?}\label{what-do-you-need-to-do}}

\begin{itemize}
\tightlist
\item
  Modes of presentation
\item
  Structure
\end{itemize}

\hypertarget{how-do-you-get-started}{%
\section{How do you get started?}\label{how-do-you-get-started}}

\hypertarget{authoring-in-word}{%
\subsection{Authoring in Word}\label{authoring-in-word}}

Copy basic idea in and then link to the Word stuff you already have

\begin{itemize}
\tightlist
\item
  Workshop
\item
  Entering equations
\end{itemize}

\hypertarget{what-about-latex}{%
\subsection{What about LaTeX?}\label{what-about-latex}}

\hypertarget{hobsons-choice}{%
\subsubsection{Hobson's choice}\label{hobsons-choice}}

\begin{itemize}
\tightlist
\item
  MathAltNotes and why you should avoid it
\item
  Other ways
\end{itemize}

\hypertarget{markdownrmarkdownbookdown}{%
\subsubsection{Markdown/RMarkdown/Bookdown}\label{markdownrmarkdownbookdown}}

\BeginKnitrBlock{theorem}
\label{thm:thm1} A labeled theorem here.
\EndKnitrBlock{theorem}

\BeginKnitrBlock{proposition}
\label{prp:prop1} A labeled theorem here.
\EndKnitrBlock{proposition}

\BeginKnitrBlock{lemma}
\label{lem:lem1} A labeled theorem here.
\EndKnitrBlock{lemma}

\BeginKnitrBlock{theorem}
\label{thm:thm2} A labeled theorem here.
\EndKnitrBlock{theorem}

\BeginKnitrBlock{corollary}
\label{cor:cor1} A labeled theorem here.
\EndKnitrBlock{corollary}

\hypertarget{work-in-progress-claverton-down}{%
\subsubsection{Work in progress: Claverton Down}\label{work-in-progress-claverton-down}}

\BeginKnitrBlock{Thought}
\protect\hypertarget{Thought:tho1}{}{ \label{tho:tho1} }A labeled theorem here.
\EndKnitrBlock{Thought}

\BeginKnitrBlock{Nugget}
\protect\hypertarget{Nugget:nug1}{}{ \label{nug:nug1} }A labeled theorem here.
\EndKnitrBlock{Nugget}

This is a test. Now go to \ref{prp:prop1}.

\hypertarget{accessible-diagrams}{%
\subsection{Accessible diagrams}\label{accessible-diagrams}}

Copy from the email and include a Desmos sine wave in case we have time

\hypertarget{further-information}{%
\section{Further information}\label{further-information}}

\begin{itemize}
\tightlist
\item
  sigma
\item
  stemenable
\end{itemize}

\hypertarget{thanks-questions}{%
\section{Thanks \& questions}\label{thanks-questions}}

\begin{itemize}
\tightlist
\item
  Where you can download this
\item
  How you can contact me
\end{itemize}

\end{document}
